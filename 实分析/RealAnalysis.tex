\documentclass{ctexart}
\usepackage{pgfplots}
\usepackage{tikz}
\usepackage{silence}
\usepackage[most]{tcolorbox}
\WarningFilter{hyperref}{}
\WarningFilter{latexfont}{}
\usepackage{tikz-cd}
\usepackage{amsmath}
\usepackage{amssymb}
\usepackage{amsthm}
\usepackage{mathrsfs}
\usepackage{titling}
\usepackage{pgfplots}
\usepackage[hidelinks]{hyperref}
\usepackage{xcolor}
\usepackage{ulem}
\usepackage{titlesec}
\usepackage[a4paper, left=2.5cm, right=2.5cm, top=2.5cm, bottom=2cm]{geometry}
\setlength{\droptitle}{-3cm}
\pgfplotsset{compat=1.18}
\pagestyle{empty}
\theoremstyle{definition}
\renewcommand{\contentsname}{}

%chktex-file 37
%chktex-file 25
%chktex-file 18
%chktex-file 17
%chktex-file 11
%chktex-file 9
%chktex-file 3

\renewcommand\proofname{\indent Pf.}

\newtheorem{remark}{注.}

%标题样式

\titleformat{\section}
{\huge\bfseries\color{gray!25!cyan}}
{\thesection}
{1em}
{}

\titleformat{\subsection}
{\large\bfseries\color{gray!25!cyan}}
{\thesubsection}
{0.5em}
{}

\newenvironment{comm}[1][]{
    \begin{tcolorbox}[
        enhanced,
        breakable,
        colframe=gray,
        coltitle=gray,
        colback=white,
        colbacktitle=white,
        boxrule=1pt,
        titlerule=-1pt,
        title={\bf #1},
    ]

}{
    \end{tcolorbox}
}

\newenvironment{cirno}[1][]{
    \begin{tcolorbox}[
        enhanced,
        breakable,
        colframe=cyan!75!gray,
        coltitle=cyan!75!gray,
        colback=white,
        colbacktitle=white,
        boxrule=1pt,
        titlerule=-1pt,
        title={\bf #1},
        overlay unbroken and last={
            \node[anchor=north east, xshift=0, yshift=0] at (frame.north east) {
                \includegraphics[width=1.2cm]{resources/琪露诺.png}
            };
        },
    ]

}{
    \end{tcolorbox}
}

\newenvironment{dashcirno}[1][]{
    \begin{tcolorbox}[
        enhanced,
        breakable,
        colframe=white,
        borderline={0.5mm}{0mm}{cyan!75!gray,dashed},
        coltitle=cyan!75!gray,
        colback=white,
        colbacktitle=white,
        boxrule=1pt,
        titlerule=-1pt,
        title={\bf #1},
        overlay unbroken and last={
            \node[anchor=north east, xshift=0, yshift=0] at (frame.north east) {
                \includegraphics[width=1.2cm]{resources/琪露诺.png}
            };
        },
    ]

}{
    \end{tcolorbox}
}

\newenvironment{dotcirno}[1][]{
    \begin{tcolorbox}[
        enhanced,
        breakable,
        colframe=white,
        borderline={0.5mm}{0mm}{cyan!75!gray,dotted},
        coltitle=cyan!75!gray,
        colback=white,
        colbacktitle=white,
        boxrule=1pt,
        titlerule=-1pt,
        title={\bf #1},
        overlay unbroken and last={
            \node[anchor=north east, xshift=0, yshift=0] at (frame.north east) {
                \includegraphics[width=1.2cm]{resources/琪露诺.png}
            };
        },
    ]

}{
    \end{tcolorbox}
}

\newcounter{theorem}[section]%=====定理=====
\renewcommand{\thetheorem}{{定理 }\thesection.\arabic{theorem}}
\newenvironment{theorem}[1][]{
    \refstepcounter{theorem}
    \begin{dashcirno}[\bf \thetheorem. #1]
    
}{
    \end{dashcirno}
}
%=====引理=====
\newcommand{\thelemma}{{引理 }\thesection.\arabic{theorem}}
\newenvironment{lemma}[1][]{
    \refstepcounter{theorem}
    \begin{dashcirno}[\bf \thelemma. #1]
    
}{
    \end{dashcirno}
}
%=====推论=====
\newcommand{\thecorollary}{{推论 }\thesection.\arabic{theorem}}
\newenvironment{corollary}[1][]{
    \refstepcounter{theorem}
    \begin{dashcirno}[\bf \thecorollary. #1]
    
}{
    \end{dashcirno}
}
%=====命题=====
\newcommand{\theproposition}{{命题 }\thesection.\arabic{theorem}}
\newenvironment{proposition}[1][]{
    \refstepcounter{theorem}
    \begin{dashcirno}[\bf \theproposition. #1]
    
}{
    \end{dashcirno}
}

\newcounter{definition}[section]%=====定义=====
\renewcommand{\thedefinition}{{定义 }\thesection.\arabic{definition}}
\newenvironment{definition}[1][]{
    \refstepcounter{definition}
    \begin{cirno}[\bf \thedefinition. #1]
    
}{
    \end{cirno}
}
\newcounter{example}[section]%=====例题=====
\renewcommand{\theexample}{{例 }\thesection.\arabic{example}}
\newenvironment{example}[1][]{
    \refstepcounter{example}
    \begin{dotcirno}[\bf \theexample. #1]
    
}{
    \end{dotcirno}
}

\newenvironment{solution}{
    \begin{proof}[Sol.]}{\end{proof}}

%=====标题=====
\title{实分析}
\author{xiaou0}

%====数学宏=====
\renewcommand{\O}{\varnothing}
\renewcommand{\L}{\mathscr{L}}
\newcommand{\C}{\mathbb{C}}
\newcommand{\R}{\mathbb{R}}
\newcommand{\Z}{\mathbb{Z}}
\newcommand{\Q}{\mathbb{Q}}
\newcommand{\set}[1]{\left\{\,#1\,\right\}}
\renewcommand{\vec}[1]{\mathbf{#1}}
\newcommand{\diam}{\operatorname{diam}}


\begin{document} %=====文档开始=====
\maketitle
\tableofcontents

\newpage

\hspace{1em}
\begin{figure}[htbp]
    \centering
    \includegraphics{resources/cirno_title.jpg}
\end{figure}
\begin{center}
    {\bf 特别鸣谢}
    
    琪露诺|幻想乡大学纯粹与应用数学博士

    https://www.pixiv.net/artworks/90486424
\end{center}
\hspace{5cm}
\begin{center}
    {\bf 作者的话}

    Ciallo~ 我是xiaou0, 一名爱好数学的普通高中生.

    该材料仅供交流学习, {\bf 仅供交流学习!} 贩卖兜售的话xiaou0会哭的TAT

    我个人的数学写作风格可能尚为稚嫩. 如果你正在学习数学, 希望这份材料能够帮助到你!
\end{center}
\hspace{5cm}
\begin{center}
    {\bf 讨论群}

    讨论群号 (QQ): 244425765
\end{center}
\newpage

\section{集合}

本章中, 我们将介绍实分析所需要的集合和拓扑的前置知识 (当然过于基本的就不介绍了:P).

\subsection{子集族}
\begin{definition}[子集族]
    对于集合$S$, 若集合$\mathcal T$中的任一元素都是$S$中的子集, 则称$\mathcal T$是$S$上的{\bf 子集族} (collection of subsets).
\end{definition}

子集族是分析学中最重要的刻画之一, 诸多内容例如拓扑, $\sigma$-代数都是子集族的构造.

例如, $S$的幂集$\wp(S)$就是$S$的一个子集族. 事实上, 幂集是最大的子集族: 显然它包含了$S$全体的子集, 任何$S$的子集族都是$\wp(S)$的子集.

\subsection{集合的序关系}
\begin{definition}[集合的序关系]
    对于任给的集族$\mathcal C$, 我们可以定义一个偏序关系. 对于$A,B\in\mathcal C$, 我们定义
    \[A\preceq B\iff A\subset B\]
    显然构成偏序关系. 为了记号方便, 也为了不引起歧义, 我们用$A\subset B$来代指$A\preceq B$.
\end{definition}
\begin{remark}
    这本书内我们用$\subset$表示子集, 用$\subsetneq$表示真子集.
\end{remark}

由定义, 下面的定理是显然的.

\begin{theorem}
    任何集合族都是偏序集.
\end{theorem}

不失一般性, 下面我们对于集合族的讨论, 总是考虑将其作为某个全集$\Omega$的子集族.

回顾上确界的定义, 在偏序集中, {\bf 上界} (upper bound)的定义为对于集合$A$中的任意元素$a$, 都有$b\ge a$的元素$b$就成为$A$的一个上界.
而{\bf 上确界} (supremum)$c$是满足对任何$A$的上界$b$, 都有$b\ge c$的那个上界$c$ (称之为上确界的{\bf 泛性质} (universal property)). 对偶地我们可以定义{\bf 下界} (lower bound)和{\bf 下确界} (infimum).

\begin{theorem}
    一个偏序集的上确界存在即唯一.
\end{theorem}
\begin{proof}
    设$m_1,m_2$都是$A$的上确界, 那么它们肯定是上界, 由上确界的泛性质可得$m_1\le m_2$和$m_2\le m_1$, 由反对称性可知$m_1=m_2$. 
\end{proof}

从而我们可以给出如下定义:

\begin{definition}[上界集]
    设$\mathcal C$是$X$的子集族, 则$\mathcal C$对于集合的序关系的上确界
    称为该子集族的{\bf 上界集} (supremum set).
\end{definition}

我们容易猜想结论:

\begin{theorem}
    设$\mathcal C$是对任意并封闭的集族, 则$\mathcal A\subset\mathcal C$的上界集就是它们的并:
    \[\sup{\mathcal A}=\bigcup_{A\in\mathcal A}{A}\]
\end{theorem}
\begin{proof}
    显然
    \[\bigcup_{A\in\mathcal A}{A}\]
    是$\mathcal A$的一个上界, 假若$B$是$\mathcal A$的一个上界, 那么对任意$A\in\mathcal A$,
    都有$A\subset B$, 从而$\bigcup_{A\in\mathcal A}{A}\subset B$.
\end{proof}
\begin{corollary}
    设$\mathcal C$是对有限并封闭的集族, 则$A_1,A_2,\ldots,A_n\in\mathcal C$的上界集就是它们的并:
    \[\sup\set{A_1,A_2,\ldots,A_n}=\bigcup_{k=1}^n{A_k}\]
\end{corollary}
\begin{corollary}
    设$\mathcal C$是对可数并封闭的集族, 则任意集列$\set{A_n}_{n=1}^\infty\subset\mathcal C$的上界集就是它们的并:
    \[\sup\set{A_n}_{n=1}^\infty=\bigcup_{k=1}^\infty{A_k}\]
\end{corollary}

对偶地, 我们可以定义下界集, 并且指出一个集族的下界集就是它们的交.

\subsection{集合列}

我们重点研究的对象是集合列 (特别是某个集合的子集列), 以及它们的极限.
\begin{definition}[集列]
    正整数集到某个集合族$\mathcal C$的映射$\Z_+\to\mathcal C$, 就称作一个{\bf 集列}.
\end{definition}

对应于实数列中的极限, 我们可以定义集列的上下极限:

\begin{definition}[集列的上极限]
    对于一列集合$\set{A_n}_{n=1}^\infty$, 其{\bf 上极限} (limit superior)定义为
    \[\limsup_{n\to\infty}{A_n}:=\inf_{n\ge1} {\sup_{k\ge n}{A_k}}=\bigcap_{n=1}^\infty{\bigcup_{k=n}^\infty{A_k}}\]
\end{definition}
\begin{definition}[集列的下极限]
    对于一列集合$\set{A_n}_{n=1}^\infty$, 其{\bf 下极限} (limit inferior)定义为
    \[\liminf_{n\to\infty}{A_n}:=\sup_{n\ge1} {\inf_{k\ge n}{A_k}}=\bigcup_{n=1}^\infty{\bigcap_{k=n}^\infty{A_k}}\]
\end{definition}

就像实数列极限里那样, 我们也可以借助上下极限定义一个集合列的极限:

\begin{definition}[集列的极限]
    对于一列集合$\set{A_n}_{n=1}^\infty$, 若其上下极限相等, 就称该集合列是{\bf 收敛} (converge)的, 并令
    \[\lim_{n\to\infty}{A_n}=\limsup_{n\to\infty}=\liminf_{n\to\infty}=
    \bigcap_{n=1}^\infty{\bigcup_{k=n}^\infty{A_k}}=\bigcup_{n=1}^\infty{\bigcap_{k=n}^\infty{A_k}}\]
    称之为该集合列的{\bf 极限} (limit).
\end{definition}

我们再来给出两类尤其重要的集列:

\begin{definition}[单调集列]
    对于集合列$\set{A_n}_{n=1}^\infty$, 若满足
    \[A_1\subset A_2\subset\cdots\subset A_n\subset\cdots\]
    则称集合列$\set{A_n}_{n=1}^\infty$是{\bf 单调上升}的, 若
    \[A_1\supset A_2\supset\cdots\supset A_n\supset\cdots\]
    则称集合列$\set{A_n}_{n=1}^\infty$是{\bf 单调下降}的, 若满足任意一者,
    就称该列是{\bf 单调} (monotonic)的.
\end{definition}

类比实数列的单调有界定理, 我们有

\begin{theorem}\label{thm:单调集合列的极限}
    若$\set{A_n}_{n=1}^\infty$是单调的, 那么其必然是收敛的. 
    若$\set{A_n}_{n=1}^\infty$是单调上升的, 则
    \[\lim_{n\to\infty}{A_n}=\bigcup_{k=1}^\infty{A_n}\]
    若$\set{A_n}_{n=1}^\infty$是单调下降的, 则
    \[\lim_{n\to\infty}{A_n}=\bigcap_{k=1}^\infty{A_n}\]
\end{theorem}
\begin{proof}
    假设$\set{A_n}_{n=1}^\infty$是单调上升的, 则
    \[\limsup_{n\to\infty}{A_n}=\bigcap_{n=1}^\infty{\bigcup_{k=n}^\infty{A_k}}\]
    由于$\set{A_n}_{n=1}^\infty$单调上升, 从而
    \[\bigcup_{k=n}^\infty{A_k}\]
    对任意$n$都相同, 因此
    \[\bigcap_{n=1}^\infty{\bigcup_{k=n}^\infty{A_k}}=\bigcup_{k=1}^\infty{A_k}\]
    同样地
    \[\liminf_{n\to\infty}{A_n}=\bigcup_{n=1}^\infty{\bigcap_{k=n}^\infty{A_k}}\]
    由于$\set{A_n}_{n=1}^\infty$单调上升, 从而
    \[\bigcap_{k=n}^\infty{A_k}=A_n\]
    于是
    \[\limsup_{n\to\infty}{A_n}=\liminf_{n\to\infty}{A_n}=\bigcup_{k=1}^\infty{A_k}\]
    类似地可以证明单调下降的情况.
\end{proof}
\begin{definition}[单调类]
    若集合$X$的子集族$\mathcal M$满足对任意单调的集列$\set{A_n}_{n=1}^\infty\subset\mathcal M$, 都有
    \[\lim_{n\to\infty}{A_n}\in\mathcal M\]
    则称$\mathcal M$是$X$的一个{\bf 单调类} (monotonic class).
\end{definition}

\newpage
\section{测度论 I}

自从小学开始学习几何以来, 我们似乎一直默许了一个概念: {\bf 面积}. 面积似乎无处不在, 人们对他司空见惯了,
却仍然没有意识到一个严重的问题, 那就是{\bf 如何在数学里严格定义面积}?

你或许会说: 这很简单啊, 比如说正方形的面积是$S=a^2$, 长方形是$S=ab$, 圆形是$S=\pi r^2$... 面积谁不会算啊, 你拿小学知识忽悠我呢?

事实上, 但凡学数学的都能看出, 目前为止我们给面积下的定义都太狭隘了. 我们只需要随手构建一个
看上去更神秘的集合, 凭我们之前的知识就束手无策了. 例如求全体有理数$\Q$在$\R$ (作为空间)中的长度.

而本章的目标就是解决面积 (或体积)概念的问题, 并引出一个更加抽象, 泛用的概念: 测度.

\subsection{$\sigma$-代数}

就像在定义实函数之前我们严格定义了实数: 既然要定义测度, 我们就需要指明哪些集合是可测的.
从我们一贯的几何直觉中, 不相交的集合的无交并的面积应该是双方面积的简单加和. 我们可以将这个概念推广到无限情况,
也就是{\bf 可数可加性}. 至于为什么没有不可数可加性, 显然对不可数个数定义代数求和并没有意义.

\begin{definition}[$\sigma$-代数]
    对于一个集合$X$, 其子集族$\Sigma$若满足:
    \begin{enumerate}
        \item[A1] $\O\in\Sigma$.
        \item[A2] $A\in\Sigma\implies X\setminus A\in\Sigma$.
        \item[A3] 对于子族$\set{A_i}_{i\in I}\subset\Sigma$, 若指标集$I$至多可数, 则$\bigcup_{i\in I}{A_i}\in\Sigma$.
    \end{enumerate}
    那么就称$\Sigma$是$X$上的一个{\bf $\sigma$-代数} ($\sigma$-algebra). 
    若要求条件A3中的$I$是有限集, 则也称$\Sigma$是一个{\bf 代数}.
\end{definition}

显然, 上述定义蕴含了以下结果:

\begin{theorem}
    设$\Sigma$是$X$上的$\sigma$-代数, 则:
    \begin{enumerate}
        \item[A4] $X\in\Sigma$.
        \item[A5] 对于子族$\set{A_i}_{i\in I}\subset\Sigma$, 若指标集$I$可数, 则$\bigcap_{i\in I}{A_i}\in\Sigma$.
    \end{enumerate}
\end{theorem}

上述两个定理是显然的, A4由A1和A2推出, A5可以由De Morgan定律 (对偶律)推出.
事实上, 如果用A4和A5分别替换A1和A3, 给出的定义和原定义还是等价的.

显然, 幂集$\wp(X)$也是$X$上的$\sigma$-代数, 我们容易证明它是$X$上最大的代数. 但我们更多情况下考虑的其实是最小的代数,
具体来说, 是包含某些特定集合的最小代数. 我们发现代数之间也有一系列明确的包含关系:

\begin{definition}[子$\sigma$-代数]
    对于$X$上的$\sigma$-代数$\Sigma$, 若$\Sigma'\subset\Sigma$也是$X$的$\sigma$-代数, 则称为$\Sigma$的{\bf 子$\sigma$-代数} (sub $\sigma$-algebra). 
\end{definition}

由于任何子集族都是$\wp(X)$的子集, 所有$\sigma$-代数都是$\wp(X)$的子代数, 这就印证了刚刚说的, $\wp(X)$在集合族之间的序关系下, $\wp(X)$就是那个最大元.
这给了我们构造最小代数的一种方法: 类似于群论中的生成子群,
既然要找包含某些特定集合的最小代数, 只需找出所有包含这些集合的代数, 取其交集就是满足泛性质的下确界.
不过在此之前, 我们需要先验证$\sigma$-代数的任意交还是$\sigma$-代数 (即确界一定存在).

\begin{theorem}\label{thm:sigma代数的交}
    设$\set{\Sigma_j}_{j\in J}$是一族$X$的$\sigma$-代数, 则
    \[\Sigma_0:=\bigcap_{j\in J}{\Sigma_j}\]
    也是$X$的$\sigma$-代数.
\end{theorem}
\begin{proof}
    我们逐一验证定义中的公理:
    \begin{enumerate}
        \item[A1] 显然对每个$\Sigma_j$, 都有$\O\in\Sigma_j$, 因此$\O\in\Sigma_0$成立.
        \item[A2] 对任意$S\in\Sigma_0$, 那么对每个$\Sigma_j$都有$S\in\Sigma_j$, 因此$X\setminus S\in\Sigma_j$, 从而$X\setminus S\in\Sigma_0$.
        \item[A3] 对任意$S_1,S_2,\ldots\in\Sigma_0$, 对于每个$\Sigma_j$都有$S_1,S_2,\ldots\in\Sigma_j$, 从而
        $\bigcup_{i=1}^\infty{S_i}\in\Sigma_j$, 从而也有$\bigcup_{i=1}^\infty{S_i}\in\Sigma_0$.
    \end{enumerate}
\end{proof}

因此我们可以作下述定义:

\begin{definition}[生成的$\sigma$-代数]
    设$\mathcal M$是$X$的一个子集族. 令
    \[\operatorname{coll}(\mathcal M):=\set{\Sigma\mid\Sigma\supset\mathcal M, \Sigma\text{为}\sigma\text{-代数}}\]
    显然其非空, 令
    \[\sigma(\mathcal M):=\bigcap_{\Sigma\in\operatorname{coll}(\mathcal M)}{\Sigma}\]
    则由\ref{thm:sigma代数的交}, $\sigma(\mathcal M)$是一个$\sigma$-代数, 称之为{\bf $\mathcal M$生成的代数} (generated by $\mathcal M$).
\end{definition}

就像我们之前讨论的一样, $\sigma(\mathcal M)$就是包含$\mathcal M$最小的$\sigma$-代数. 我们还可以定义两个$\sigma$-代数的积构造,
这也是为了积空间的内容作铺垫.

\begin{definition}[积$\sigma$-代数]
    设$\Sigma_1,\Sigma_2$分别是$X_1,X_2$的$\sigma$-代数. 则
    \[\mathcal R:=\set{S_1\times S_2\mid S_1\in\Sigma_1,S_2\in\Sigma_2}\]
    并记$\Sigma_1\otimes\Sigma_2=\sigma(\mathcal R)$, 它是$X_1\times X_2$上的$\sigma$-代数, 称为$\Sigma_1,\Sigma_2$
    的{\bf 积$\sigma$-代数} (product). 有时为了区别它和Cartesian积, 也称之为两个$\sigma$-代数的张量积.
\end{definition}

上面的定义和积拓扑的定义也有异曲同工之妙. 两者的过程都类似于先构造出空间中的矩形, 再通过矩形生成完整的拓扑 ($\sigma$-代数).
这是因为有些集合不一定能完全分解为两个集合的积.

\begin{theorem}
    如果$X$上的代数$\mathcal M$是一个单调类, 则$\mathcal M$是$\sigma$-代数.
\end{theorem}
\begin{proof}
    设$\mathcal M$是代数, 则对任意有限并封闭. 对于其中任何一列集合$\set{A_n}_{n=1}^\infty$, 其并集
    可取为极限
    \[\lim_{n\to\infty}{\bigcup_{i=1}^n{A_n}}\]
    显然这个列是单调的. 由单调类定义, 其极限也在$\mathcal M$中, 这就证明了$\mathcal M$是$\sigma$-代数.
\end{proof}

\subsection{可测空间}

可测空间, 其实就是可以定义测度的空间. 需要注意的是, 这和测度空间是两个概念: 测度空间是配备了特定测度的空间.

\begin{definition}[可测空间]
    指定了$\sigma$-代数$\Sigma$的集合$X$, 称为一个{\bf 可测空间} (measurable space), 通常记作二元组$(X,\Sigma)$.
    此时称$S\in\Sigma$是$X$的一个{\bf 可测集} (measurable set).
\end{definition}

可测空间的定义是非常简单的, 同样地我们也可以定义其积对象:

\begin{definition}[积可测空间]
    对于两个可测空间$(X,\mathcal A)$和$(Y,\mathcal B)$, 那么$\mathcal A\otimes\mathcal B$是
    $X\times Y$上的$\sigma$-代数, 记
    \[(X\times Y,\mathcal A\otimes\mathcal B)\]
    为这两个可测空间的{\bf 积} (product), 通常简记作$X\times Y$.
\end{definition}

\subsection{可测映射}

用范畴论的语言, 全体可测空间构成可测空间范畴$\mathsf{Meas}$, 这个范畴中的态射就是可测映射, 类似于拓扑学中的连续映射, 它的定义也非常相似:
\begin{definition}[可测映射]
    设$(X,\mathcal A)$, $(Y,\mathcal B)$是两个可测空间, 映射$f:X\to Y$若满足对任意$S\in\mathcal B$, 其原象都满足$f^{-1}(S)\in\mathcal A$,
    则称$f$是$X$到$Y$的一个{\bf 可测映射} (measurable map). 一个可测空间$(X,\Sigma)$上的全体可测映射组成的集合记作$\L^0(X,\Sigma)$.
\end{definition}

就像拓扑学中的连续函数一样, 我们也有如下结论:

\begin{theorem}
    可测映射的复合仍然是可测映射.
\end{theorem}
\begin{proof}
    设$f:X\to Y$, $g:Y\to Z$是复合映射, 则$g\circ f:X\to Z$满足, 对任意$Z$中的可测集$S$, $g^{-1}(S)$是$Y$中可测集,
    从而$(g\circ f)^{-1}(S)$是$X$中可测集.
\end{proof}

\begin{theorem}
    一个映射$f:X\to Y_1\times Y_2\times\cdots\times Y_n$可测, 当且仅当每一个分量函数$f_i=\pi_i\circ f$可测.
    \[\large\begin{tikzcd}[row sep=huge]
        X & {Y_i} \\
        {\prod_jY_j}
        \arrow["{\pi_i\circ f}", dashed, from=1-1, to=1-2]
        \arrow["f"', from=1-1, to=2-1]
        \arrow["{\pi_i}"', from=2-1, to=1-2]
    \end{tikzcd}\]
\end{theorem}
\begin{proof}
    由于$f$可测, 对任意$Y_1\times Y_2\times\cdots\times Y_n$的可测集$S$, $f^{-1}(S)$是$X$中的可测集. 由定义
    \[\mathcal R=\set{S_1\times S_2\times\cdots\times S_n\mid S_1\in\Sigma_1,S_2\in\Sigma_2,\ldots,S_n\in\Sigma_n}\]
    其中$\Sigma_1,\Sigma_2,\ldots,\Sigma_n$分别是$Y_1,Y_2,\ldots,Y_n$所配备的$\sigma$-代数.
    从而任何$Y_1\times Y_2\times\cdots\times Y_n$中的可测集都可以表示为可数个$\mathcal R$中元素的并集,
    而每个$\mathcal R$中元素在$Y_i$上投影的原像都是$Y_i$的可测集 (积的泛性质), 可测集的可数并依然可测, 从而每个分量函数是可测的.

    另一方面, 若每个分量函数都可测, 那么证明是类似的. 对于$\mathcal R$中的每个元素, 其在$Y_i$上的原像都是可测集, 那么$f$自然可测.
\end{proof}

\subsection{$[0,\infty]$上的代数运算}

在处理测度的时候, 我们难免会遇到$\infty$, 一个理由是我们希望赋予无限大的集以一个测度, 例如$\R$就有无限长.
同样一个序列的极限或级数的和也很可能是$\infty$, 如果规避讨论$\infty$, 会导致很多麻烦.
在测度论的语境中, 我们一般会做出一些特殊规定:

\begin{enumerate}
    \item[加法] 对于$0\le a\le\infty$, 我们总是规定$a+\infty=\infty+a=\infty$. 
    \item[乘法] 对于$0<a\le\infty$, 我们定义$a\cdot\infty=\infty\cdot a=\infty$; 特别地, $0\cdot\infty=\infty\cdot0=0$. 
\end{enumerate}

这么定义可能会显得很奇怪, 因为它似乎和数学分析中的结论有出入, 尤其是$0\cdot\infty=0$, 在数学分析中我们知道这是一个不定式, 不可以视作定值
\sout{凭什么$0\cdot\infty=0$啊! galgame里不是这样的!}, 然而容易验证, 这样定义的运算的交换律和结合律在$[0,\infty]$上都是自动成立的.



\subsection{测度}

现在一切铺垫都准备就绪, 我们给出一般情况下测度的定义:

\begin{definition}[测度]
    设$(X,\Sigma)$是可测空间, 若定义在其$\sigma$-代数上的集函数$\mu:\Sigma\to\R_{\ge0}\cup\set{\infty}$满足:
    \begin{enumerate}
        \item[M1] $\mu(\O)=0$.
        \item[M2] (可数可加性) 设$\set{E_i}_{i\in I}\subset\Sigma$是两两不交的可测集族, 且指标集$I$至多可数, 则
        \[\mu\left(\bigcup_{i\in I}{E_i}\right)=\sum_{i\in I}{\mu(E_i)}\]
    \end{enumerate}
    则称$\mu$为$(X,\Sigma)$上的一个{\bf (非负)测度} (measure). 指定了测度的可测空间$(X,\Sigma,\mu)$称为{\bf 测度空间} (measure space).
\end{definition}
\begin{remark}
    显然$\mu(E_i)$一定是大于等于0的正数. 因此是正项级数, 重排并不会改变它的敛散性和极限.
\end{remark}
\begin{remark}
    我们并没有要求测度一定是有限数, 测度可以取$\infty$. 这类集合也是容易构造的.
\end{remark}

\begin{theorem}[测度的保序性]
    对于配备了测度$\mu$的可测空间$(X,\Sigma)$, 对于$A,B\in\Sigma$, 总有
    \[A\subset B\implies\mu(A)\le\mu(B)\]
\end{theorem}
\begin{proof}
    若$A\subset B$, 则$B\setminus A\in\Sigma$, 从而$\mu(B\setminus A)\ge 0$. 由可数可加性,
    \[\mu(B)=\mu(A)+\mu(B\setminus A)\]
    这就证明了定理.
\end{proof}
\begin{theorem}[测度的连续性]\label{thm:测度的连续性}
    对于配备了测度$\mu$的可测空间$(X,\Sigma)$, 若可测集合列$\set{A_n}_{n=1}^\infty\subset\Sigma$单调上升或单调下降, 且$\mu(A_1)<\infty$, 则其极限可测并且
    \[\lim_{n\to\infty}{\mu(A_n)}=\mu\left(\lim_{n\to\infty}{A_n}\right)\]
\end{theorem}
\begin{proof}
    我们分别证明两种情况:
    \begin{enumerate}
        \item 假设$\set{A_n}_{n=1}^\infty$是单调上升的, 由\ref{thm:单调集合列的极限},
        \[\lim_{n\to\infty}{A_n}=\bigcup_{n=1}^\infty{A_n}\]
        我们作$B_1=A_1$, 对于$n>1$, $B_n=A_n\setminus A_{n-1}$. 显然
        \[\bigcup_{n=1}^\infty{A_n}=\bigcup_{n=1}^\infty{B_n}\]
        我们只是为$\set{A_n}_{n=1}^\infty$作了一次去重, 显然由可数可加性
        \[\mu\left(\bigcup_{n=1}^\infty{B_n}\right)=\sum_{n=1}^\infty{\mu(B_n)}=\mu(A_1)+\sum_{n=2}^\infty{\mu(A_n\setminus A_{n-1})}=\mu(A_1)+\sum_{n=2}^\infty{\left(\mu(A_n)-\mu(A_{n-1})\right)}\]
        等式右边的级数是一个望远镜级数, 其和为$\mu\left(\lim_{n\to\infty}{A_n}\right)-\mu(A_1)$, 于是
        \[\mu\left(\bigcup_{n=1}^\infty{B_n}\right)=\mu\left(\lim_{n\to\infty}{A_n}\right)\]
        这就证明了$\set{A_n}_{n=1}^\infty$单调上升的情况.
        \item 假设$\set{A_n}_{n=1}^\infty$是单调下降的, 我们令$B_n$为差分, 对任意$n$, 定义
        \[B_n=A_n\setminus A_{n+1}\]
        容易证明
        \[A_1\setminus\bigcap_{n=1}^\infty{A_n}=\bigcup_{n=1}^\infty{B_n}\]
        显然$\set{B_n}_{n=1}^\infty$是两两不交的, 显然有
        \[\mu\left(A_1\setminus\bigcap_{n=1}^\infty{A_n}\right)=\sum_{n=1}^\infty{\mu(B_n)}=\sum_{n=1}^\infty{\left(\mu(A_n)-\mu(A_{n+1})\right)}\]
        由 (1)中类似的方法可以得到
        \[\mu\left(\bigcap_{n=1}^\infty{A_n}\right)=\lim_{n\to\infty}{\mu(A_n)}\]
        这就证明了$\set{A_n}_{n=1}^\infty$单调上升的情况.
    \end{enumerate}
\end{proof}

\begin{theorem}[Fatou引理, 集合形式]
    对于配备了测度$\mu$的可测空间$(X,\Sigma)$, 以及可测集合列$\set{A_n}_{n=1}^\infty\subset\Sigma$, 总有
    \[\mu\left(\liminf_{n\to\infty}{A_n}\right)\le\liminf_{n\to\infty}{\mu(A_n)}\]
    若存在可测集$A_0$使得$\mu(A_0)<\infty$且对任意$n$, $A_n\subset A_0$, 则有
    \[\mu\left(\limsup_{n\to\infty}{A_n}\right)\ge\limsup_{n\to\infty}{\mu(A_n)}\]
\end{theorem}

\begin{proof}
    考虑$\set{A_n}_{n=1}^\infty$的下极限, 显然有
    \[\liminf_{n\to\infty}{A_n}=\bigcup_{n=1}^\infty{\bigcap_{k=n}^\infty{A_k}}\]
    显然$\bigcap_{k=n}^\infty{A_k}$是随$n$单调上升的. 令
    \[B_1=\bigcap_{k=1}^\infty{A_k},\quad B_n=\bigcap_{k=n}^\infty{A_k}\setminus\bigcap_{k=n-1}^\infty{A_k}\]
    那么
    \[\begin{aligned}
        &\mu\left(\liminf_{n\to\infty}{A_n}\right)=\mu\left(\bigcup_{n=1}^\infty{B_n}\right)=\sum_{n=1}^\infty{\mu(B_n)}\\
        =&\mu\left(\bigcap_{k=1}^\infty{A_k}\right)+\sum_{n=2}^\infty{\left(\mu\left(\bigcap_{k=n}^\infty{A_k}\right)-\mu\left(\bigcap_{k=n-1}^\infty{A_k}\right)\right)}\\
        =&\lim_{n\to\infty}{\mu\left(\bigcap_{k=n}^\infty{A_k}\right)}\le\liminf_{n\to\infty}{\mu(A_n)}
    \end{aligned}\]
    
    同样地, 考虑$\set{A_n}_{n=1}^\infty$的上极限, 显然有
    \[\limsup_{n\to\infty}{A_n}=\bigcap_{n=1}^\infty{\bigcup_{k=n}^\infty{A_k}}\]
    考虑
    \[A_0\setminus\left(\limsup_{n\to\infty}{A_n}\right)\]
    其中$A_0$是包含全体$A_n$的有限测度集. 由De Morgan律, 
    \[A_0\setminus\left(\limsup_{n\to\infty}{A_n}\right)=A_0\setminus\bigcap_{n=1}^\infty{\bigcup_{k=n}^\infty{A_k}}
    =\bigcup_{n=1}^\infty{\left(A_0\setminus\bigcup_{k=n}^\infty{A_k}\right)}
    =\bigcup_{n=1}^\infty{\bigcap_{k=n}^\infty{(A_0\setminus A_k)}}=\liminf_{n\to\infty}{(A_0\setminus A_n)}\]
    从而
    \[\limsup_{n\to\infty}{A_n}=A_0\setminus\left(\liminf_{n\to\infty}{(A_0\setminus A_n)}\right)\]
    于是
    \[\mu\left(\limsup_{n\to\infty}{A_n}\right)=\mu(A_0)-\mu\left(\liminf_{n\to\infty}{(A_0\setminus A_n)}\right)\ge\mu(A_0)-\liminf_{n\to\infty}{(\mu(A_0)-\mu(A_n))}=\limsup_{n\to\infty}{\mu(A_n)}\]
\end{proof}

我们再引入一种有趣的测度:

\begin{definition}[计数测度]
    对于集合$X$, 我们在$\wp(X)$上定义测度$\mu_c$满足:
    \begin{enumerate}
        \item 若$A\subset X$是有限集, 则$\mu_c(A)=|A|$.
        \item 若$A\subset X$是无限集, 则$\mu_c(A)=\infty$.
    \end{enumerate}
    容易验证这是一个测度, 称之为集合$X$上的{\bf 计数测度} (counting measure).
\end{definition}

\subsection{外测度}

从本节开始我们开始尝试定义$\R$上的一般测度, 也就是Lebesgue测度, 也就是我们熟悉的长度.
虽然我们已经有了测度的概念, 但是定义Lebesgue测度也不总是一件容易的事情. 外测度就是为了解决这类问题而存在的,
我们先不考虑测度的性质, 而是从更宽松的条件入手:

\begin{definition}[外测度]
    设集合$X$的幂集$\wp(X)$上的集函数$\mu^*:\wp(X)\to\R_{\ge0}\cup\set{\infty}$若满足:
    \begin{enumerate}
        \item[OM1] $\mu^*(\O)=0$.
        \item[OM2] (单调性) $A\subset B\implies \mu^*(A)\le\mu^*(B)$.
        \item[OM3] (可数次可加性) 对于$X$的任意子集的至多可数族$\set{E_n}\subset\wp(X)$, 其并集的外测度不大于每个集合的外测度之和:
        \[\mu^*\left(\bigcup_{n=1}^\infty{E_n}\right)\le\sum_{n=1}^\infty{\mu^*(E_n)}\]
    \end{enumerate}
    则称$\mu^*$是$X$上的一个{\bf 外测度} (outer measure).
\end{definition}
\begin{remark}
    注意外测度定义域一定是全体子集.
\end{remark}

\subsection{Lebesgue外测度}

我们可以尝试在实线$\R$上定义一个简单的外测度, 既然如此, 我们自然先从最简单的情况入手:
显然对于$\R$上的一个{\bf 区间}, 例如$(0,1),(0,1],[0,1]$. 我们定义他们的长度 (外测度)就是
它们端点之间的距离, 即
\[\diam{(a,b)}=b-a\]
这个记号源自度量空间中的{\bf 直径}.

\begin{definition}[Lebesgue外测度]
    我们定义$\R$的一个子集$S$的{\bf Lebesgue外测度}为使用可数个区间覆盖该子集的区间长度的下确界, 即
    \[m^*(S):=\inf\set{\sum_{j\in J}{\diam{I_j}}\mid \bigcup_{j\in J}{I_j}\supset S}\]
    其中每个$I_j$都是区间, 且指标集$J$为可数集.
\end{definition}

\begin{proposition}
    $m^*:\wp(\R)\to\R_{\ge0}\cup\set{\infty}$是一个外测度. 
\end{proposition}
\begin{proof}
    我们分别证明外测度的几条公理:
    \begin{enumerate}
        \item[OM1] 显然空集可以任意取一个长度趋近于$0$的一列区间, 证明$m^*(\O)=0$的过程是平凡的.
        \item[OM2] 假若$A\subset B\subset\R$, 那么覆盖$B$的覆盖也一定覆盖$A$, 于是若$m^*(A)>m^*(B)$必构成矛盾.
        \item[OM3] 显然当对每个$n$, $\set{I_{nj}}$覆盖$E_n$时, 
        \[\bigcup_{n=1}^\infty{\bigcup_{j=1}^\infty{I_{nj}}}\]
        也覆盖
        \[\bigcup_{n=1}^\infty{E_n}\]
        类似的可以证明
        \[m^*\left(\bigcup_{n=1}^\infty{E_n}\right)>\sum_{n=1}^\infty{m^*(E_n)}\]
        不可能成立.        
    \end{enumerate}
\end{proof}

我们现在定义了一个看上去很像测度的东西, 但是{\bf 它并不满足可数可加性}, 对于这个事实有一个非常经典的反例的构造,
我们会在之后介绍.

\subsection{Carathéodory条件}

我们现在知道: 有一些病态的集合使得外测度不满足可数可加性. 那么我们要做的事情也很简单: 只要把这些\sout{坏东西}不干净的集合去掉, 我们就能得到一个测度.

但是, 如何判断一个集合是不是病态的? 德国数学家Constantin Carathéodory (1873--1950)给出了这个问题的答案, 他给出并证明了一个集合是病态的条件, 也就是Carathéodory条件:

\begin{comm}[Carathéodory条件]
    设$X$上有外测度$\mu^*$, 对于$E\subset X$, 若对任意$A\subset X$都有
    \[\mu^*(A)=\mu^*(A\cap E)+\mu^*(A\setminus E)\]
    即对任意集合$A$, 其测度等于其在$E$之内的部分与$E$之外的部分之和. 则称集合$E$满足{\bf Carathéodory条件}.
    简称C-条件.
\end{comm}

Carathéodory也证明了这个条件的合理性, 即这个条件的的确确把所有病态集合都剔除了, 也就是重要的Carathéodory定理:

\begin{theorem}[Carathéodory定理]\label{thm:Carathéodory定理}
    设集合$X$以及集合$X$上的外测度$\mu^*$, 那么:
    \begin{enumerate}
        \item $X$上全体满足C-条件的集合构成一个$\sigma$-代数$\Sigma$.
        \item 将外测度$\mu^*$限制在$\Sigma$上的限制映射$\mu^*|_\Sigma$是一个测度.
    \end{enumerate}
\end{theorem}
\begin{proof}
    先证明其满足$\sigma$-代数.
    \begin{enumerate}
        \item[A1] 显然对于$\O$, 对任意$A$都满足$A\cap\O=\O$, $A\setminus\O=A$, 从而$\O$满足C-条件, 这就证明了$\O\in\Sigma$.
        \item[A2] 任取$E\in\Sigma$, 则$E$满足C-条件, 令$F=X\setminus E$, 则对任意$A\subset X$, 都有
        \[A\cap F=A\setminus E,\quad A\setminus F=A\cap E\]
        从而
        \[\mu^*(A\cap F)+\mu^*(A\setminus F)=\mu^*(A\cap E)+\mu^*(A\setminus E)=\mu^*(A)\]
        于是$X\setminus E\in\Sigma$.
        \item[A3] 先证明二元情况. 设$A,B\in\Sigma$, 要证$A\cup B\in\Sigma$. 由于$A\in\Sigma$, 由C-条件, 对任给的$S\in X$, 有
        \[\mu^*(S)=\mu^*(S\cap A)+\mu^*(S\setminus A)\tag{I}\]
        由于$B$也满足C-条件, 代入$S'=S\setminus A$有
        \[\mu^*(S')=\mu^*(S'\cap B)+\mu^*(S'\setminus B)\tag{II}\]
        将 (II)代入 (I), 有
        \[\mu^*(S)=\mu^*(S\cap A)+\mu^*(S\setminus A\cap B)+\mu^*(S\setminus A\setminus B)\tag{III}\]
        我们调整一下上式的形式, 由于$S\setminus A\setminus B=S\setminus(A\cup B)$, 记$C=A\cup B$.
        由于$S\cap C=S\cap(A\cup B)=(S\cap A)\cup(S\cap B)=(S\cap A)\cup(S\setminus A\cap B)$, 从而
        \[\mu^*(S\cap C)=\mu^*((S\cap A)\cup(S\setminus A\cap B))\le\mu^*(S\cap A)+\mu^*(S\setminus A\cap B)\tag{IV}\]
        将 (IV)代回 (III)就立即得到了
        \[\mu^*(S)\ge\mu^*(S\cap C)+\mu^*(S\setminus C)\tag{V.I}\]
        又由于$(S\cap C)\cup(S\setminus C)=S$, 由外测度的公理OM3可得
        \[\mu^*(S)\le\mu^*(S\cap C)+\mu^*(S\setminus C)\tag{V.II}\]
        结合 (V.I), (V.II)就证明了
        \[\mu^*(S)=\mu^*(S\cap C)+\mu^*(S\setminus C)\]
        从而$C=A\cup B\in\Sigma$, 由归纳法容易证明有限情况. 并且假若$A\cap B=\O$, 那么
        \[\mu^*(A\cup B)=\mu^*(A\cup B\cap B)+\mu^*(A\cup B\setminus B)=\mu^*(B)+\mu^*(A)\]
        这证明了$\mu^*|_\Sigma$是有限可加的. 下面证明可数情况, 设$\set{E_i}_{i=1}^\infty$是一列集合. 不失一般性, 
        我们假设其是两两无交的. 若非如此, 可取$E'_n=E_n\setminus\bigcup_{i=1}^{n-1}{E_i}$, 且对于每个$i$都有$E_i\in\Sigma$. 令
        \[E=\bigcup_{i=1}^\infty{E_i}\]
        我们要证明$E\in\Sigma$. 不妨设对正整数$n$,
        \[H_n=\bigcup_{i=1}^n{E_i}\]
        我们已经证明了$H_n$满足C-条件, 对任意集合$S\subset X$都有
        \[\mu^*(S)=\mu^*(S\cap H_n)+\mu^*(S\setminus H_n)\tag{VI}\]
        由于$H_n\subset E$, 从而$X\setminus E\subset X\setminus H_n$, 由外测度单调性OM2得
        \[\mu^*(S\setminus H_n)\ge\mu^*(S\setminus E)\]
        代入 (VI)有
        \[\mu^*(S)\ge\mu^*(S\cap H_n)+\mu^*(S\setminus E)\tag{VII}\]
        注意到$S\cap H_n$是$S\cap E_1,S\cap E_2,\ldots,S\cap E_n$的无交并, 因此由已知结论 (有限可加)
        \[\mu^*(S\cap H_n)=\sum_{i=1}^n{\mu^*(S\cap E_i)}\tag{VIII}\]
        联立 (VII), (VIII)有
        \[\mu^*(S)\ge\sum_{i=1}^n{\mu^*(S\cap E_i)}+\mu^*(S\setminus E)\]
        令上式$n\to\infty$, 由实数列极限保序性有
        \[\mu^*(S)\ge\sum_{i=1}^\infty{\mu^*(S\cap E_i)}+\mu^*(S\setminus E)
        \ge\mu^*\left(\bigcup_{i=1}^\infty{S\cap E_i}\right)+\mu^*(S\setminus E)=\mu^*(S\cap E)+\mu^*(S\setminus E)\tag{IX}\]
        同样地, 由于$(S\cap E)\cup(S\setminus E)=S$, 由外测度的公理
        \[\mu^*(S)\le\mu^*(S\cap E)+\mu^*(S\setminus E)\tag{X}\]
        联立 (IX), (X)有
        \[\mu^*(S)=\mu^*(S\cap E)+\mu^*(S\setminus E)\]
        这就证明了$E\in\Sigma$. 再对有限可加性取极限容易证明可数可加性. 于是我们证明了$\mu^*|_\Sigma$是一个测度.
    \end{enumerate}
\end{proof}

借助Carathéodory定理定义的$\mu^*|_\Sigma$称为{\bf 外测度$\mu^*$诱导的测度}, Carathéodory定理指出, 每个外测度都能导出一个测度.

\subsection{Lebesgue测度}

\begin{definition}[Lebesgue测度]
    令$\mathcal M$为$\R$中全体满足Carathéodory条件的子集构成的族, 令$m=m^*|_\mathcal{M}$是
    Lebesgue外测度在$\mathcal M$上的限制, 由\ref{thm:Carathéodory定理}, $(\R,\mathcal M,m)$构成一个测度空间, 称测度$m$为
    {\bf Lebesgue测度} (Lebesgue measure). 当我们提到$\R$时, 若没有明确指定, 默认使用该测度作为测度.
\end{definition}

至此, 我们完成了$\R$上长度的定义, 下面我们来分析它的一些性质:

\begin{theorem}[平移不变性]
    设$A$是Lebesgue可测集, 则对任意$k\in\R$, 有
    \[m(A)=m(k+A)\]
\end{theorem}
\begin{proof}
    易证区间平移后仍然是等长的区间, 且平移后的全体区间依然覆盖平移后的集合, 因此平移后的测度不会大于原本的测度, 即
    \[m^*(E+k)\le m^*(E)\]
    只需再次套用一遍上式,
    \[m^*(E-k'+k')\le m^*(E-k')\implies m^*(E)\le m^*(E-k')\]
    由于$k$是任意的, 只需任选$k'=-k$即可, 从而结论成立.
\end{proof}

\begin{example}[单点集]
    求$m\set{0}$.
\end{example}
\begin{solution}
    显然$(-1/n,1/n)$是一列包含$\set{0}$的区间列, 其单调下降且测度趋于$0$, 从而$\boxed{m\set{0}=0}$.
\end{solution}

\begin{example}[有理数集$\Q$]
    求$m(\Q)$.
\end{example}
\begin{solution}
    显然$\Q$可以写作无交并
    \[\Q=\bigcup_{q\in\Q}{\set{q}}\]
    由上例的结论以及平移不变性, 我们容易得出所有$m\set{q}=0$, 从而
    \[\boxed{m(\Q)=\sum_{q\in\Q}{m\set{q}}=\sum_{q\in\Q}{0}=0}\]
\end{solution}

\begin{figure}[htbp]
    \centering
    \begin{tikzpicture}[scale=0.4]
        \draw[line width=2pt] (-9,4)--(9,4) node[left,pos=0] {$A_0$};
        \draw[line width=2pt] (-9,3)--(-3,3) node[left,pos=0] {$A_1$};
        \draw[line width=2pt] (3,3)--(9,3);
        \draw[line width=2pt] (-9,2)--(-7,2) node[left,pos=0] {$A_2$};
        \draw[line width=2pt] (3,2)--(5,2);
        \draw[line width=2pt] (-5,2)--(-3,2);
        \draw[line width=2pt] (7,2)--(9,2);
        \draw[line width=2pt] (-9,1)--(-8.33,1) node[left,pos=0] {$A_3$};
        \draw[line width=2pt] (3,1)--(3.66,1);
        \draw[line width=2pt] (-5,1)--(-4.33,1);
        \draw[line width=2pt] (7,1)--(7.66,1);
        \draw[line width=2pt] (-7.66,1)--(-7,1);
        \draw[line width=2pt] (4.33,1)--(5,1);
        \draw[line width=2pt] (-3.66,1)--(-3,1);
        \draw[line width=2pt] (8.33,1)--(9,1);
    \end{tikzpicture}
    \caption{Cantor三分集}
\end{figure}

\begin{example}[Cantor三分集]
    考虑一个如下构造的集合:
    \begin{enumerate}
        \item[0.] $A_0=[0,1]$.
        \item[1.] $A_1=[0,1/3]\cup[2/3,1]$.
        \item[2.] $A_1=[0,1/9]\cup[2/9,1/3]\cup[2/3,7/9]\cup[8/9,1]$.
        \item[n.] 反复循环, 每次去掉每个区间的中间三分之一. 
        \item[$\infty$.] 目标集合$C$.
    \end{enumerate}
    具体地, $A_n$定义为
    \[A_n=A_{n-1}-\bigcup_{k=0}^\infty{\left(\frac{3k+1}{3^n},\frac{3k+2}{3^n}\right)}\]
    取其交
    \[C=\bigcap_{n=1}^\infty{A_n}\]
    称之为{\bf Cantor三分集}. 求$m(C)$.
\end{example}
\begin{solution}
    显然每个$A_n$都是$A_{n-1}$的子集, $\set{A_n}_{n=1}^\infty$是单调下降的. 并且对于每个$A_n$, 
    \[m(A_n)=\left(\frac{2}{3}\right)^n\]
    显然成立. 由\ref{thm:测度的连续性} (测度的连续性),
    \[\boxed{m(C)=m\left(\lim_{n\to\infty}{A_n}\right)=\lim_{n\to\infty}{m(A_n)}=\lim_{n\to\infty}{\left(\frac{2}{3}\right)^n}=0}\]
\end{solution}

\subsection{Vitali集}

我们将在本节指出, 并不是实数集的每个子集都是Lebesgue可测 (满足Carathéodory条件)的.

我们先在$[0,1]$上定义一个等价关系$\sim$:
\[x\sim y\iff x-y\in\Q\]
这显然是等价关系, 因为
\begin{enumerate}
    \item $x-x=0\in\Q$.
    \item $y-x=-(x-y)\in\Q$.
    \item $x-y,y-z\in \Q\implies x-z=(x-y)+(y-z)\in\Q$.
\end{enumerate}
因此可以给出$[0,1]$的一个划分$[0,1] /\sim$, 其每个元素都是两两无交的等价类$[x]$, 即
\[[x]=(x+\Q)\cap[0,1]\]
并且每个等价类$[x]$非空. 下面, 我们要应用选择公理,
从$[0,1]/\sim$中的每个等价类中选取一个代表元构成集合$\mathcal V$, 即
\[\mathcal V=\set{p[x]\mid [x]\in[0,1] /\sim}\]
其中$p$是选择函数, 由选择公理确保其存在性. 我们选取的集合$\mathcal V$就称为{\bf Vitali集}.

显然, Vitali集是不可数的. 这是因为每个等价类$[x]$是可数的, 然而$[0,1]$不可数, 而$\mathcal V$与全体$[x]$等势.

\begin{theorem}
    Vitali集$\mathcal V$是不可测的.
\end{theorem}
\begin{proof}
    我们令$\Q^\circ=\Q\cap[-1,1]$, 定义$\mathcal V$的一个平移:
    \[\mathcal V_q=\mathcal V+q=\set{\,v+q\mid v\in\mathcal V\,},\quad q\in\Q^\circ\]
    任取$p,q\in\Q^\circ$且$p\ne q$, 假设$\mathcal V_p\cap\mathcal V_q\ne\O$, 则存在$z$使得
    \[z=v_1+p\quad\land\quad z=v_2+q\quad(v_1,v_2\in\mathcal V)\]
    则$v_1-v_2=q-p\in\Q$. 从而$v_1\sim v_2$, 然而由$\mathcal V$的构造, 每个等价类中的元素唯一, 从而$v_1=v_2$, 这导致$p=q$, 与假设矛盾. 从而$\mathcal V_p,\mathcal V_q$无交, 由于$p,q$是任意的, 因此$\set{\mathcal V_q}_{q\in\Q^\circ}$是一个两两不交的可数集合族. 于是我们作它们的可数并
    \[\mathcal V^\circ=\bigcup_{q\in\Q^\circ}{\mathcal V_q}\]
    则:
    \begin{enumerate}
        \item $[0,1]\subset\mathcal V^\circ$, 这是因为$\mathcal V$对应的商集构成$[0,1]$的划分: 任给$x\in[0,1]$, 都可以表示为$v+q$, 其中$v\in\mathcal V, q\in\Q^\circ$.
        \item $\mathcal V^\circ\subset[-1,2]$, 这是因为$\mathcal V\subset[0,1]$且$\Q^\circ\subset[-1,1]$.
    \end{enumerate}
    不妨设Vitali集可测, 由Lebesgue测度的平移不变性, 对任意$q\in\Q^\circ$, $\mathcal V_q$也是可测的, 且$m(\mathcal V_q)=m(\mathcal V)$. 并且由可数可加性以及$\mathcal V^\circ$的定义,
    \[m(\mathcal V^\circ)=\sum_{q\in\Q^\circ}{m(\mathcal V_q)}\tag{I}\]
    根据测度的单调性, 
    \[1=m[0,1]\le m(\mathcal V^\circ)\le m[-1,2]=3\tag{II}\]
    若$m(\mathcal V)=0$, 则
    \[m(\mathcal V^\circ)=m\left(\bigcup_{q\in\Q^\circ}{\mathcal{V}_q}\right)=\sum_{q\in\Q^\circ}{m(\mathcal V_q)}=\sum_{q\in\Q^\circ}{m(\mathcal V)}=\sum_{q\in\Q^\circ}0=0\]
    这是由绝对收敛级数重排得到的, 与 (II)中$m(\mathcal V^\circ)\ge 1$矛盾. 若$m(\mathcal V)>0$, 则
    \[m(\mathcal V^\circ)=m\left(\bigcup_{q\in\Q^\circ}{\mathcal{V}_q}\right)=\sum_{q\in\Q^\circ}{m(\mathcal V_q)}=\sum_{q\in\Q^\circ}{m(\mathcal V)}=\infty\]
    与 (II)中$m(\mathcal V^\circ)\le 3$矛盾. 从而假设不成立. $\mathcal V$是Lebesgue不可测的.
\end{proof}

\subsection{乘积测度}

我们已经推广了$\R$上长度的概念, 现在我们可以很容易地定义$\R^d$上体积的概念.

\begin{definition}[乘积测度]
    设测度空间$(X,\mathcal{M},\mu)$和$(Y,\mathcal{N},\nu)$, 我们定义测度$\mu\times\nu:\mathcal{M}\otimes\mathcal{N}\to\R_{\ge 0}\cup\set{\infty}$, 满足
    \[(\mu\times\nu)^*(A\times B)=\mu(A)\cdot\nu(B),\quad A\in\mathcal{M}, B\in\mathcal{N}\]
    其中每个形如$A\times B$的集合称之为一个矩形. 我们可以仿照外测度的方式定义外测度
    \[(\mu\times\nu)^*(U)=\inf\set{\sum_{j\in J}{(\mu\times\nu)^*(R_j)}\mid\bigcup_{j\in J}{R_j}\supset S}\]
    其中每个$R_j$都是矩形且$J$是至多可数的, 容易证明这是一个外测度, 由Carathéodory定理, 可以确定唯一的测度$\mu\times\nu$,
    称之为$\mu$和$\nu$的{\bf 积测度} (product measure).
\end{definition}

这样我们可以推广$\R^d$上的测度, 一般情况下, 我们选取$m^d=\overset{\#d}{\overbrace{m\times m\times\cdots\times m}}$作为$\R^d$的标准测度.

\subsection{零测集}

零测集就像加法中的0, 它本身也是一个非常有用的研究对象. 因为我们已经见识了测度为0的集合不一定是空集.

\begin{definition}[零测集]
    在测度空间$(X,\Sigma,\mu)$中, 若$N\in\Sigma$满足$\mu(N)=0$, 就称$N$是一个{\bf 零测集} (null set). 
\end{definition}

显然有以下定理:

\begin{theorem}
    零测集的可数并是零测集.
\end{theorem}
\begin{proof}
    显然由测度的可数次可加性, 对于一族零测集$\set{N_i}_{i=1}^\infty$.
    \[\mu\left(\bigcup_{i=1}^\infty{N_i}\right)\le\sum_{i=1}^\infty{\mu(E_i)}=0\]
    由非负性立即得出其为零测集.
\end{proof}

\begin{definition}[几乎处处]
    对于测度空间$(X,\Sigma,\mu)$以及一个关于$X$中的点$x\in X$的命题$P$,
    我们称命题$P$在$X$上{\bf 几乎处处} (almost everywhere)成立, 当且仅当$X$去除某个零测集后该命题在任意点上成立. 常常简写作a.e.
    
    几乎处处是依赖测度的, 若有多个测度, 我们用记号 a.e. [$\mu$]来表示对测度$\mu$几乎处处成立.
\end{definition}

\subsection{特征函数}

\begin{definition}[特征函数]
    设$A\subset X$, $A$在$X$中的{\bf 特征函数} (characteristic function)是指函数$\chi_A:X\to\R$, 定义为
    \[\chi_A(x)=\begin{cases}
        1&x\in A\\
        0&x\notin A
    \end{cases}\]
    这个定义容易推广到$\C$上. 故不多赘述.
\end{definition}

特征函数是最最简单的一类函数. 它的取值就类似于一个判断, 判断一个点是否在集合内.

\begin{proposition}
    显然, 对于至多可数集族$\set{A_n}_{n=1}^\infty$, 有
    \[x\in\bigcap_{n=1}^\infty{A_n}\iff \prod_{n=1}^\infty{\chi_{A_n}(x)}=1\]
\end{proposition}
\begin{proof}
    证明是显然的, 若前 (或后)成立, 都能推出$\chi_{A_n}(x)$恒为1, 再推出另一者.
\end{proof}

一般情况下, 在实分析语境中, 我们使用符号$\chi$就默认为特征函数.

\subsection{简单函数}

在定义一般函数的积分之前, 我们可以先从最简单的情况入手. 简单函数顾名思义就是形式最简单的函数: 它的取值只可能是有限个.

\begin{definition}[简单函数]
    若函数$s:X\to Y$的像$s(X)$是一个有限集, 则称$s$为一个{\bf 简单函数} (simple function).
\end{definition}

\begin{remark}
    虽然简单函数只能取有限值, 但我们并不要求取相同值的区域必须连通. 相反, 它可以分散, 甚至可以非常分散, 只要其值域是有限集.
\end{remark}

在实分析的语境下, 我们经常要求其为可测简单函数, 也就是说, 对于每一个取值$\alpha$, 它对应的原像$s^{-1}(\alpha)$一定是一个可测集.

\begin{theorem}[典范形式]
    设$\set{\alpha_1,\alpha_2,\ldots,\alpha_n}$是可测简单函数$s:X\to\R$的全体取值, 则
    \[s=\sum_{i=1}^n{\alpha_i\chi_{s^{-1}(\alpha_i)}}\]
    称之为$s$的典范形式, 该定理容易推广到$\C$.
\end{theorem}
\begin{proof}
    证明也是显然的, 对任意$x\in X$, 若$s(x)=\alpha_k$, 则$x\in s^{-1}(\alpha_k)$, 那么$\chi_{s^{-1}(\alpha_k)}(x)=1$,
    并且对任意$\alpha_j\ne \alpha_k$, $\chi_{s^{-1}(\alpha_j)}(x)=0$, 从而
    \[\sum_{i=1}^n{\alpha_i\chi_{s^{-1}(\alpha_i)}}(x)=\alpha_k\cdot 1=\alpha_k\]
\end{proof}

\newpage
\section{空间与收敛}

在介绍积分理论之前, 我们需要一些前置概念和结论. 在分析学中, 空间就是最重要的研究对象, 也是一切的基础.
本章中就重点研究分析学中常见的空间.

我们不会对于每个空间进行特别深入的研究, 若感兴趣可以深入研究对应的课题.

\subsection{线性空间}

线性空间是一个代数的概念, 其描述的是一个具备数乘概念的Abel群.

\begin{definition}[线性空间]
    {\bf 线性空间} (vector space)是域作用在Abel群上构成的代数结构, 具体地说, 对于集合$V$和域$\mathbb K$, 定义加法
    \[+:V\times V\to V\]
    和数乘作用
    \[\cdot:\mathbb K\times V\to V\]
    并且对任意$\mathbf{a},\mathbf{b},\mathbf{c}\in V;\, k,l\in\mathbb{K}$满足:
    \begin{enumerate}
        \item[G1] (加法结合律) $(\vec a+\vec b)+\vec c=\vec a+(\vec b+\vec c)$.
        \item[G2] (零元) $\exists!\vec 0\in V,\vec 0+\vec a=\vec a+\vec 0=\vec a$.
        \item[G3] (负元) $\forall\vec a\in V,\exists!-\vec a\in V,\vec a+(-\vec a)=(-\vec a)+\vec a=\vec 0$.
        \item[Ab] (加法交换律) $\vec a+\vec b=\vec b+\vec a$.
        \item[D1] (数乘对向量加法的分配律) $k\cdot(\vec a+\vec b)=k\cdot\vec a+k\cdot\vec b$.
        \item[D2] (数乘对域加法的分配律) $(k+l)\cdot\vec a=k\cdot\vec a+l\cdot\vec a$.
        \item[A1] (乘法作用) $(k\cdot l)\cdot\vec a=k\cdot(l\cdot\vec a)$.
        \item[A2] (单位元作用) $\exists!1\in\mathbb K, 1\cdot\vec a=\vec a$.
    \end{enumerate}
    就称$(V,+,\cdot,\mathbb K)$是一个{\bf 线性空间} (vector space).
\end{definition}

线性空间的研究是线性代数的主题, 在此我们只给出一些简单定义:

\begin{definition}[线性算子]
    设$V,W$是线性空间, 映射$f:V\to W$若对任意$v\in V,w\in W$都满足
    \[f(k\cdot v+l\cdot w)=k\cdot f(v)+l\cdot f(w)\]
    则称$f$为一个{\bf 线性算子} (linear operator).
\end{definition}

一般地说, 我们把一个空间映射到另一个空间的操作就称为一个{\bf 算子} (operator). 线性算子在有限维空间$\R^n\to\R^m$中对应$P^{n\times m}$的矩阵.

\begin{definition}[线性同构]
    若线性算子$f$是双射, 则称$f$为一个{\bf 线性同构} (isomorphism), 称存在线性同构$f:V\to W$的空间$V,W$是{\bf 同构} (isomorphic)的.
\end{definition}

\subsection{拓扑空间}

拓扑空间是拓扑学的研究对象. 实分析需要非常深厚的拓扑学功底, 因此在学习实分析前最好先学习点集拓扑学,
这里给出一些非常基本的拓扑学定义:

\begin{definition}[拓扑空间]
    设集合$X$和$X$的子集族$\mathcal T$. 若$\mathcal T$满足:
    \begin{enumerate}
        \item[Tp1] $\O\in\mathcal T$, $X\in\mathcal T$.
        \item[Tp2] (任意并封闭) 任给一族$\set{U_\alpha}_{\alpha\in J}\subset\mathcal T$, 都有
        \[\bigcup_{\alpha\in J}{U_\alpha}\in\mathcal T\]
        \item[Tp3] (有限交封闭) 任给有限个$U_1,\ldots,U_n\in\mathcal T$, 都有
        \[\bigcap_{i=1}^n{U_i}\in\mathcal T\] 
    \end{enumerate}
    则称$\mathcal T$是$X$的一个{\bf 拓扑} (topology), 指定了拓扑的集合$(X,\mathcal T)$称为一个
    {\bf 拓扑空间} (topological space).
\end{definition}

本质上说, 拓扑空间研究的就是那些邻域, 并用集合去刻画点与点之间的相邻关系. 拓扑学中,
我们注重研究两类最重要的集合:

\begin{definition}[开集]
    拓扑空间$(X,\mathcal T)$中, $\mathcal T$中的集合称为一个{\bf 开集} (open set).
\end{definition}

它的对偶概念:

\begin{definition}[闭集]
    拓扑空间$(X,\mathcal T)$中, 若$X\setminus C$是一个开集, 则称$C$是一个{\bf 闭集} (closed set).
\end{definition}

显然有以下定理:

\begin{theorem}
    设$X$是拓扑空间, 则:
    \begin{enumerate}
        \item[Tp4] $\O$是闭集, $X$也是闭集.
        \item[Tp5] 有限个闭集的并也是闭集.
        \item[Tp6] 闭集的任意交也是闭集.
    \end{enumerate}
\end{theorem}

\begin{proof}
    我们逐一证明每个结论:
    \begin{enumerate}
        \item[Tp4] 显然$X\setminus X=\O$, $X\setminus\O=X$. 
        \item[Tp5] 设$C_1,\ldots,C_n$是有限个闭集, 则$X\setminus C_1,\ldots,X\setminus C_n$是有限个开集,
        由De Morgan律,
        \[\bigcup_{i=1}^n{C_i}=X\setminus\bigcap_{i=1}^n{X\setminus C_i}\]
        \item[Tp6] 同样地, 由De Morgan律
        \[\bigcap_{\alpha\in J}{C_\alpha}=X\setminus\bigcup_{\alpha\in J}{X\setminus C_\alpha}\] 
    \end{enumerate}
\end{proof}

事实上, 用Tp4, Tp5, Tp6替换Tp1, Tp2, Tp3可以得到一个完全对偶的拓扑空间, 但一般我们采用开集定义.

\begin{definition}[邻域]
    拓扑空间$X$中, 包含点$x$的一个开集$U$称为$x$的一个{\bf 邻域} (neighborhood).
\end{definition}

内部和闭包也是常见的两种集合的运算:

\begin{definition}[内部和闭包]
    设$X$是拓扑空间, $A\subset X$是子集, $A$的{\bf 内部} (interior)是包含于$A$的全体开集的并 (上确界),
    即被$A$包含的最大开集, 即
    \[\dot{A}:=\sup\set{\text{开集} O\mid O\subset A}\]
    $A$的{\bf 闭包} (closure)是包含$A$的
    全体闭集的交 (下确界), 即包含$A$的最小闭集, 即
    \[\overline{A}:=\inf\set{\text{闭集} K\mid K\supset A}\]
\end{definition}

更多内容可以参考关于点集拓扑学的教材, 个人比较推荐James.R.Munkres的Topology. (等我有空了再写一本点集拓扑)

\subsection{连续映射}

数学分析中, 我们已经初步定义了$\R^m\to\R^n$中的连续性, 这个概念可以进行推广: 事实上,
所谓连续就是保持点相邻关系的映射, 在拓扑的语言里刻画如下:

\begin{definition}[连续映射]
    设$X,Y$是拓扑空间, 若映射$f:X\to Y$满足对任意开集$U\subset Y$, 其原像
    $f^{-1}(U)\subset X$是$X$中的开集, 那么称$f$是{\bf 连续的} (continuous).
\end{definition}

例如在$\R\to\R$的情况下, $\R$上的拓扑是由全体开区间生成的, 那么连续性就是说对任意 (小)的开集$U$ (例如 $(x_0-\epsilon,x_0+\epsilon)$),
映射到这个集合的所有点在$\R$中也是开集. 不过我们更熟悉的定义应该是:

\begin{definition}[在一点处连续]
    设$X,Y$是拓扑空间, $f:X\to Y$是映射, $x\in X$, 若对任意$f(x)\in Y$的邻域$V\ni f(x)$,
    都有$x$的邻域$x\in U\subset X$, 使得$f(U)\subset f(V)$, 那么就称$f${\bf 在点$x$处连续} (continuos at $x$).
\end{definition}

这个定义套用到$\R\to\R$上, 就是我们熟悉的连续性的定义.

\begin{theorem}
    设$X,Y$是拓扑空间, $f:X\to Y$是映射, 那么下列命题等价:
    \begin{enumerate}
        \item $f$连续.
        \item 对任意$X$的子集$A$, 都有$f(\overline{A})\subset\overline{f(A)}$.
        \item 对任意$Y$中闭集$K$, 都有$f^{-1}(K)$是$X$中闭集.
        \item $f$在$X$的每一点处连续.
    \end{enumerate}
\end{theorem}

上述定理的证明较为平凡, 并且不在本书主题范围内, 故略过 (感兴趣可以自己证).

\subsection{Hausdorff空间}

\begin{definition}[Hausdorff空间]
    对于拓扑空间$X$, 若对于任意两个点$x_1,x_2\in X$, 总存在$x_1,x_2$的邻域$U_1,U_2$使得$U_1\cap U_2=\O$.
    那么就称$X$是一个{\bf Hausdorff空间} (Hausdorff space).
\end{definition}
%\subsection{紧致性}
%\subsection{度量空间}
%\subsection{网和收敛※}
%\subsection{完备性}
%\subsection{点态收敛}
%\subsection{一致收敛}
%\subsection{紧致收敛※}
%\subsection{拓扑线性空间}
%\subsection{凸性}
%\subsection{赋范空间}
%\subsection{Banach空间}
%\subsection{Fréchet微分※}
%\subsection{内积空间}
%\subsection{Hilbert空间}

\end{document}